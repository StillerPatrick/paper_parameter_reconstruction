\documentclass{article}
\usepackage[german]{babel}
\usepackage[utf8]{inputenc}
\usepackage{graphicx}


\begin{document}


\section{Aufgabenstellung}
Die Kleinwinkelstreuung in Englisch Small Angle X-Ray Scattering (SAXS) ist eine universelle Technik zur Untersuchung von Feststoffen. Dabei liefert SAXS Informationen über die kristalline Struktur, chemische Komposition und die physikalische Eigenschaften des untersuchten Feststoffes \cite{THEORYSAXS}. Die Untersuchung von Feststoffen unter Einfluss von Kurzpulslasern ist eine wichtige Aufgabe der heutigen Physik. Wissen über das verhalten von Feststoffen unter  Einfluss von Kurzpulslasern kann offene Fragen in der Krebsforschung und in der Astrophysik beantworten\cite{SAXS18}. In diesem Beleg wurde SAXS für die Untersuchung von Plasma eingesetzt. 

\begin{figure}[h]
	\centering
	\includegraphics [scale=0.4]{images/saxs_setup.png}
	\caption{Schematischer Aufbau des Experimentaufbaus. Im rechten Teil der Abbildung sind Detektorbilder in Abhängigkeit des Einsetzten des   		Röntgenlaserpulses dargestellt. Bildquelle: \cite{SAXS18}}
	\label{fig:saxssetup}
\end{figure}



\end{document}